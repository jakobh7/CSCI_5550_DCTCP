% ****** Start of file DataCenterTCP_Writeup.tex ******
%
%   This file is part of the APS files in the REVTeX 4.2 distribution.
%   Version 4.2a of REVTeX, December 2014
%
%   Copyright (c) 2014 The American Physical Society.
%
%   See the REVTeX 4 README file for restrictions and more information.
%
% TeX'ing this file requires that you have AMS-LaTeX 2.0 installed
% as well as the rest of the prerequisites for REVTeX 4.2
%
% See the REVTeX 4 README file
% It also requires running BibTeX. The commands are as follows:
%
%  1)  latex apssamp.tex
%  2)  bibtex apssamp
%  3)  latex apssamp.tex
%  4)  latex apssamp.tex
%
\documentclass[%
% reprint,
%superscriptaddress,
%groupedaddress,
%unsortedaddress,
%runinaddress,
%frontmatterverbose,
%preprint,
%preprintnumbers,
%nofootinbib,
%nobibnotes,
%bibnotes,
amsmath,amssymb,
aps,
%pra,
%prb,
%rmp,
%prstab,
%prstper,
%floatfix,
]{revtex4-2}

\usepackage{graphicx}% Include figure files
\usepackage{dcolumn}% Align table columns on decimal point
\usepackage{bm}% bold math
\usepackage{hyperref}% add hypertext capabilities



\begin{document}

\title{DCTCP:\\Congestion Control In a Datacenter Network Structure}

\author{Jakob Horner}

\author{Mohammad Hadi}


\date{\today}

\begin{abstract}
\begin{description}
\item[Abstract]
In data center network topologies, there are often strict expectations on throughput and round trip time even when dealing with large amounts of data. With normal network message passing approaches, often the large and varied flows of a data center can break the implemented congestion control systems. We attempt to implement a TCP over UDP protocol that offers congestion control explicitly tailored to data center topologies based on previous data center TCP research. This protocol is tested on a virtual data center network made with Mininet. 

The git repo can be accessed here: \url{https://github.com/jakobh7/CSCI_5550_DCTCP}
\end{description}
\end{abstract}

\maketitle

\section{Introduction}
In the past decade, with the increase of social networks, cloud computing, and consolidation of  other data into single data centers, it has become extremely important to provide consistent and improved performance. While there have been different strategies employed to address this performance, the implementation of a specific data center TCP protocol is a lucrative prospect that can offer cost savings for data center owners by offering greater performance without upgrading hardware.

The topology of data centers is distinct from common network topologies. Often data centers are created with a "fat-tree" network topology to separate and balance workloads. This "fat-tree" topology has a top layer switch connected to distinct lower-level switches which connect to server racks to process data. These network topologies often use a Partition/Aggregate workflow pattern to split work across switches in a layer and receive information in a real-time fashion. Because of the layered structure of the tree, and the workflow dependency on the workers below low latency is required for every level of the tree so that the overall response can be in real-time. This difference in topology and performance expectation makes it reasonable to create a specialized approach to improve performance.

The previous research on data center networks show that data center network traffic is primarily TCP 



\end{document}
%
% ****** End of file DataCenterTCP_Writeup.tex ******
